% LaTeX Article Template - customizing page format
%
% LaTeX document uses 10-point fonts by default.  To use
% 11-point or 12-point fonts, use \documentclass[11pt]{article}
% or \documentclass[12pt]{article}.
\documentclass{article}

% Set left margin - The default is 1 inch, so the following 
% command sets a 1.25-inch left margin.
\setlength{\oddsidemargin}{0.25in}

% Set width of the text - What is left will be the right margin.
% In this case, right margin is 8.5in - 1.25in - 6in = 1.25in.
\setlength{\textwidth}{6in}

% Set top margin - The default is 1 inch, so the following 
% command sets a 0.75-inch top margin.
\setlength{\topmargin}{-0.25in}

% Set height of the text - What is left will be the bottom margin.
% In this case, bottom margin is 11in - 0.75in - 9.5in = 0.75in
\setlength{\textheight}{8in}
\usepackage{fancyhdr, lastpage}
\usepackage{tikz}
\usepackage{amsmath,amssymb,amsthm}
\usetikzlibrary{calc}
\usepackage{enumitem}
\usepackage{soul}
\usetikzlibrary{positioning}
\graphicspath{ {./} }
\setlength{\parskip}{5pt} 
\pagestyle{fancyplain}

% Universes
\newcommand{\NN}{\mathbb{N}}
\newcommand{\ZZ}{\mathbb{Z}}
\newcommand{\QQ}{\mathbb{Q}}
\newcommand{\RR}{\mathbb{R}}
\newcommand{\CC}{\mathbb{C}}

% Groups commands
\newcommand{\inv}{^{-1}}
\newcommand{\lcm}{\mathrm{lcm}}
\newcommand{\lr}[1]{\langle #1 \rangle}
\newcommand{\Inn}{\mathrm{Inn}}
\newcommand{\iso}{\cong}
\newcommand{\normal}{\triangleleft}
% Set the beginning of a LaTeX document
\begin{document}

\lhead{Drew Remmenga MATH 458}
\rhead{HW \#8}
%\lhead{Independent Study}
%\rhead{R Lab}


\begin{enumerate}

   \item Prove proposition 6.6 from the Module 6 notes.
    
        \textbf{Proposition 6.6} (Properties of Multiplication in Rings). Let $a,b,c$ belong to a ring $R$. Then 
            \begin{enumerate}
                \item $a0=0a=0$ 
To prove $a0=0$ Since $a0=(0+0)a \implies a0=a0+a0$ by the left distributive law. Implies $0+a0=a0 +a0$ since 0 is the additive identity so $0+a0=a0$ hence $a0=0$. Now for $0a=0$. $0a =(0+0)a \implies 0+0a=0a+0a$ by the right distributive law. Hence $0=0a$ by cancellation. 
                \item $a(-b)=(-a)(b)=-(ab)$
Since a0=0, a(b+(-b))=0 since -b is the additive inverse of b. ab+a(-b)=0 by the left distributive law. a(-b) is then the additive inverse of ab therefore a(-b)=-ab. b(-a)=-ab follows from without loss of generality of a and b. 
                \item $(-a)(-b)=(ab)$
Since a0=0 if R is a ring then $-a\in R$ so we have (-a)0=0. Then we have -a(b+(-b))=0 because b is the additive inverse of -b. Implies -ab+(-a)(-b)=0 by the left distributive law. (-a)(-b) is the additive inverse of -(ab) then. As (ab) is the additive inverse of -(ab) we havve that additive inverses are unique so we can say (-a)(-b) = ab.
                \item $a(b-c)=ab-ac$ and $(b-c)a=ba-ca$
     a(b-c)=a(b+(-c))=ab+a(-c) by the left distributive law. Implies this is equal to ab+-(ac) using part b). So ab-ac is equivelence. Similarly (b-c)a=(b+(-c))a=ba+(-c)a right distributive law. =ba-ca using part b) so ba+(-c)a=ba-ca.
                
            \hspace{-2em}If $R$ has a unity element $1$, then 
                \item $(-1)a=-a$
Consider (-1)a+a=-1a+1a as 1a =a. Then byu the right distributive law -1a+a=(-1+1)a=-1a+a=0a then as a0=0 we have (-1+1)a=0 so as -a is the additive inverse of a -a=(-1)a.
                \item $(-1)(-1)=1$
From part c) we have (-a)(-b)=(ab) therefore (-1)(-1)=1*1=1
                \vspace*{2em}
            \end{enumerate}

     \item Suppose that $a$ and $b$ belong to a commutative ring $R$ with unity. If $a$ is a unit and $b^2=0$, show that $a+b$ is a unit.
     Consider $(a-b)(a+b)=a^{2}-b^{2}$ as R is a commutative ring. Since $b^{2}=0$ we have $(a+b)(a-b)=a^{2}+0$ SDince a is a unit we can write $(a+b)(a^{-1}-ba^{-2})=1$ so $a+b$ is an invertible element. Hence a+b is a unit. 

     \item The set $\RR[x]$ of all polynomials in the variable $x$ with real coefficients under ordinary addition and multiplication is a commutative ring.
     \begin{enumerate}
        \item What is unity in $\RR[x]$? What are the units of $\RR[x]$? Explain. 
        f(x)=1 and f(x)=-1. Unity in $\RR[x]$ means $I\in R$ such that $Ir=r=rI \forall r \in R$.

        \item Show that $\ZZ[x]$ forms a subring of $R$, where $\ZZ[x]$ is the subset of $\RR[x]$ with integer coefficients. 
        Let $a\in \ZZ[x]$ and $b \in \ZZ[x]$ then $a-b \in \ZZ[x]$ because the integers are closed under subtraction. Now consider $ab$. The product of integers is always an integer so $ab\in\ZZ[x]$. So it is closed under multiplication. Thus $\ZZ[x]$ forms a subring. 

     \end{enumerate}


    \item An element $a$ in a ring $R$ with unity is called \emph{nilpotent} if there exists a positive integer $n$ such that $a^n=0$. 
    \begin{enumerate}
        \item Give an example of a nontrivial ring $R$ and a nonzero nilpotent element $a$.
        $R=\ZZ_{4}$ where $a=2$.

        \item Show that for an arbitrary ring $R$ with unity, if $a$ is a nilpotent element of $R$, then $1-a$ is a unit. (Hint: Consider $(1-a)(1+a+a^2+\cdots + a^{n-1})$.)
        Let $a$ be a nilpoint element in an arbitrary ring $R$ with unity. If the index of $a$ is $n$ then $a^{n}=0$ but $a^r \neq 0$ for $r<n$ Now $(1+a+a^2+\cdots + a^{n-1})=frac{1-a^{n}}{1-a}$. As $1\in R$ and $a\in R$ then $(1+a+a^2+\cdots + a^{n-1})\in R$ so $(1-a)(1+a+a^2+\cdots + a^{n-1})=1-a^{n}$ so $(1-a)$ is a unit. Here $(1+a+a^2+\cdots + a^{n-1})$ is the inverse of $(1-a)$.

        \item Show that for a \emph{commutative} ring $R$ with unity, the set of nilpotent elements forms a subring.
Let $S$ be the set of all nilpotent elements of a commutative ring $R$ with unity. Let $a,b \in S$ So $a^{m}=b^{n}=0$ for some $m,n\in \ZZ$. Then $a+b \in S$ since $(a+b)^{m+n}=0$. And $ab\in S$ since $(ab)^{min(m,n)} =0$. Therefore $S$ is a subring of $R$.
        \end{enumerate}
   \item Let $R$ and $S$ be commutative rings. Prove that $(a,b)$ is a zero-divisor in $R\oplus S$ if and only if $a$ or $b$ is a zero-divisor or exactly one of $a$ or $b$ is 0. 
Let $(a,b)$ be a zero divisor of $R\oplus S$. Then there exists a nonzero element $(c,d),c\in R,d\in S$ such that $(ac,bd)=(0,0)$. Case 1: $c\neq0 d\neq 0$.. If $a=0,b\neq0\implies bd =0$ so b is a zero divisor. If $a\neq 0, b=0 \implies ac=0$ so a is a zero divisor. If $a\neq 0,b\neq 0\implies ac=0, bd=0$ hence a and b are zero divisors. Case 2: WLOG $c\neq0, d=0$. First  if $a=0,b\neq0\implies bd =0$ so b is a zero divisor. If $a\neq 0, b=0 \implies (ac,bd)=(0,0)$. If $a\neq 0,b\neq 0\implies bd=0$ hence b is a zero divisor. Case 3 is case 2 wlog $c=0,d\neq0$. Now for the converse. Let a be a zero divisor. Then $\exists c\neq 0$ such that ac=0. Now (a,b)(c,d)=(0,0) so (a,b) is a zero divisor. Then wlog consider b as a zero divisor. then $\exists d \neq 0$ such that (a,b)(0,d)=(0,0) so (a,b) is a zero divisor. Now for the third case let exactly one of a and b be zero. Then (a,b)(c,0)=(0,0) when a=0. Now wlog consider when b=0. Then (a,b)(0,d)=(0,0). In all cases it follows that (a,b) is a zero divisor. 
\end{enumerate}
\end{document}