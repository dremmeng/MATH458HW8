\documentclass{article}

\usepackage[margin=1in]{geometry}
\usepackage{fancyhdr, lastpage}
\usepackage{tikz}
\usepackage{amsmath,amssymb,amsthm}
\usetikzlibrary{calc}
\usepackage{enumitem}
\usepackage{soul}
\usepackage{multicol}

% Universes
\newcommand{\NN}{\mathbb{N}}
\newcommand{\ZZ}{\mathbb{Z}}
\newcommand{\QQ}{\mathbb{Q}}
\newcommand{\RR}{\mathbb{R}}
\newcommand{\CC}{\mathbb{C}}

% Groups commands
\newcommand{\inv}{^{-1}}
\newcommand{\lcm}{\mathrm{lcm}}
\newcommand{\lr}[1]{\langle #1 \rangle}
\newcommand{\Inn}{\mathrm{Inn}}
\newcommand{\iso}{\cong}
\newcommand{\normal}{\triangleleft}


%%%%%%%%%%%%%%%%%%%%%%%%%%%%%%%%%%%%%%%%%%%%%%%%%%%%%%%%%%%%%%
\setlength{\parindent}{0cm}
\pagestyle{fancy}
\lhead{MATH458 Abstract Algebra}
\rhead{Homework 8}

%%%%%%%%%%%%%%%%%%%%%%%%%%%%%%%%%%%%%%%%%%%%%%%%%%%%%%%%%%%%%%
\begin{document}
\section*{Homework 8}

Unless otherwise indicated, you must justify all answers/steps. See the Canvas assignment for more information about the homework requirements. 

\begin{enumerate}
    \item Prove proposition 6.6 from the Module 6 notes.
    
        \textbf{Proposition 6.6} (Properties of Multiplication in Rings). Let $a,b,c$ belong to a ring $R$. Then 
        \begin{multicols}{2}
            \begin{enumerate}
                \item $a0=0a=0$
                \item $a(-b)=(-a)(b)=-(ab)$
                \item $(-a)(-b)=(ab)$
                \item $a(b-c)=ab-ac$ and $(b-c)a=ba-ca$
                
            \columnbreak
                
            \hspace{-2em}If $R$ has a unity element $1$, then 
                \item $(-1)a=-a$
                \item $(-1)(-1)=1$
                \vspace*{2em}
            \end{enumerate}
        \end{multicols}

     \item Suppose that $a$ and $b$ belong to a commutative ring $R$ with unity. If $a$ is a unit and $b^2=0$, show that $a+b$ is a unit.
     

     \item The set $\RR[x]$ of all polynomials in the variable $x$ with real coefficients under ordinary addition and multiplication is a commutative ring.
     \begin{enumerate}
        \item What is unity in $\RR[x]$? What are the units of $\RR[x]$? Explain. 
        

        \item Show that $\ZZ[x]$ forms a subring of $R$, where $\ZZ[x]$ is the subset of $\RR[x]$ with integer coefficients. 
        

     \end{enumerate}


    \item An element $a$ in a ring $R$ with unity is called \emph{nilpotent} if there exists a positive integer $n$ such that $a^n=0$. 
    \begin{enumerate}
        \item Give an example of a nontrivial ring $R$ and a nonzero nilpotent element $a$.
        

        \item Show that for an arbitrary ring $R$ with unity, if $a$ is a nilpotent element of $R$, then $1-a$ is a unit. (Hint: Consider $(1-a)(1+a+a^2+\cdots + a^{n-1})$.)
        

        \item Show that for a \emph{commutative} ring $R$ with unity, the set of nilpotent elements forms a subring.
        

        
    \end{enumerate}


    \item Let $R$ and $S$ be commutative rings. Prove that $(a,b)$ is a zero-divisor in $R\oplus S$ if and only if $a$ or $b$ is a zero-divisor or exactly one of $a$ or $b$ is 0. 
    
\end{enumerate}
\end{document}